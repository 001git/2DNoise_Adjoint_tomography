% this is <DL1recording.tex>
% ----------------------------------------------------------------------------
% $Id$
% 
% Copyright (c) 2008 by Thomas Forbriger (BFO Schiltach) 
% 
% Recording data from the DL1 data logger
% 
% REVISIONS and CHANGES 
%    22/12/2008   V1.0   Thomas Forbriger
%    15/10/2010   V1.1   comment on message() versus match() filter in
%                        syslog-ng
%    18/04/2011   V1.2   added logrotate documentation
%    01/04/2014   V1.3   update to current version of DL1logger
% 
% ============================================================================
%
\documentclass[twoside]{article}
%\usepackage{ngerman}
\usepackage{pslatex}
\usepackage{anysize}
\usepackage{amsmath}
%\usepackage[slantedgreek]{mathtime}
\usepackage{graphicx}
\usepackage{textfit}
\usepackage{fancyhdr}
\usepackage{pepigerm}
\usepackage{booktabs}
\usepackage{multicol}
\usepackage{verbatim}
\usepackage[bf,small]{caption}
\usepackage{xspace}
\bibliographystyle{pepigerm}
%----------------------------------------------------------------------
\newcommand{\mytitle}{Recording data from the Thies DL1 pluviometer}
%----------------------------------------------------------------------
\marginsize{2.5cm}{1.5cm}{1.5cm}{1.5cm}
\pagestyle{fancy}
\rhead[\mytitle]{\thepage}
\lhead[\thepage]{\mytitle}
\chead{}
\rfoot[Black Forest Observatory, \today]{Thomas Forbriger}
\lfoot[Thomas Forbriger]{Black Forest Observatory, \today}
\cfoot{}
\sloppy
\columnsep20pt
\parindent0pt
\parskip 10pt
%======================================================================
% macros
\newcommand{\DTDL}{Thies DL1 data logger}
\newcommand{\Dlaunchscript}{\texttt{launchDL1logger.sh}}
\newcommand{\Dlogscript}{\texttt{DL1logger.sh}}
\newcommand{\Dntpscript}{\texttt{ntpreport.sh}}
\newcommand{\DLdirect}{\texttt{DL1direct}}
\newcommand{\DLlogger}{\texttt{DL1logger}}
\newcommand{\Dloggerhost}{\textsc{flocke}}
\newcommand{\DloggerhostIP}{\texttt{192.168.1.34}}
\newcommand{\Dstoragehost}{\textsc{pinatubo}}
\newcommand{\DstoragehostIP}{\texttt{192.168.1.17}}
\newcommand{\Ddisplayhost}{\textsc{stromboli}}
\newcommand{\DdisplayhostIP}{\texttt{192.168.1.41}}
\newcommand{\Dserialdevice}{\texttt{/dev/ttyUSB0}}
\newcommand{\Dloggeruser}{\texttt{dl1}}
\newcommand{\Dloggeruserhome}{\texttt{/home/dl1}}
\newcommand{\Dchannel}{\texttt{WR1}}
%======================================================================
\begin{document}
\mbox{ }\par
\vspace{-1.5cm}
%\hrule
\medskip
{\noindent\Large\textbf{\mytitle}\vspace{-14pt}\par}
%\smallskip\hrule
\rule{\textwidth}{0.5pt}
%\bigskip
\par
\vspace{-\parskip}
Technical Documentation
\hfill
\begin{raggedleft}
Thomas Forbriger\\
Black Forest Observatory (BFO)\\
Heubach 206,
D-77709 Wolfach\\
\vspace{3pt}
\today\ (\mbox{\texttt{$$Revision$$}})\\
\input{version.xxx}
\par
\end{raggedleft}
\bigskip
\thispagestyle{plain}
%-------------------------------------------------------------
\begin{multicols}{2}
\parindent10pt
\parskip 0pt
\tableofcontents
%\listoftables
%\listoffigures
\parindent0pt
\parskip 10pt
%======================================================================
\section{Pluviometer data acquisition}
\subsection{Hardware components}
The rain gauge (a tiping bucket) in front of the BFO laboratory building
produces an eletrical pulse for each 0.1\,mm of precipitation.
These pulses are counted by a \DTDL.
The \DTDL\ maintains a data ring-buffer.
Its contents can be accessed through the local display, a memory card, or the
RS-232 serial port.
We use a Linux host (\Dloggerhost) connected to the serial port to control
the \DTDL\ and to collect precipitation data.

\subsection{Data acquisition program}
Two programs are currently available to control the \DTDL.
They are \DLdirect\ (Section~\ref{sec:prog:dldirect}) and \DLlogger\
(Section~\ref{sec:prog:dllogger}).
The first supports manual control of the \DTDL\ and should not be used while
the second is in operation.
The second (\DLlogger) is the actual data acquisition program.
It controls the \DTDL, synchronizes its internal clock to the system clock of
the logger host (\Dloggerhost), and maintains an archive of data.
See Section~\ref{sec:prog:dllogger} for a detailed description in particular
for information on data sampling, poll cycles, message logging, and so on.

The data acquisition program (\DLlogger) is started from a shell script
(\Dlogscript, see Section~\ref{sec:prog:dlogscript}).
This shell script sets the correct patterns for data file path names
and initializes the serial port prior to execution of \DLlogger.
The script \Dlogscript\ itself is executed from the launch-script
\Dlaunchscript\ (see Section~\ref{sec:prog:dlaunchscript}).
The latter first checks whether an instance of the logger is in operation. 
If not, the logger is executed in background operation mode.
The launch-script \Dlaunchscript\ should be executed periodically by the cron
deamon. 
See Section~\ref{sec:conf:crontab:logger} for an example of an entry in the
crontab.
This ensures that the data acquisition is resumed automatically upon
unexpected termination of the acquisition program.

\subsection{Diagnostics}
The data acquisition program writes its diagnostic messages to the system's
syslog facility. 
The messages are tagged \DLlogger.
See Section~\ref{sec:prog:dllogger} for information on log levels used for
different diagnostic messages.
A simple configuration that maintains local publically readable log files
is given in Section~\ref{sec:conf:syslog:logger}.
This configuration also sends messages to a second system for monitoring
purposes.
Two files are maintained. 
One contains all messages that are tagged \DLlogger.
These messages comprise also purely informational messages that are issued in
each poll cycle to indicate the activity of \DLlogger. 
This log stream is also exported to the display host \Ddisplayhost\ for
monitoring purposes.
A second log file contains only messages with level notice and higher (all
levels except level info). 
This log file is meant to be archived together with the data to support error
analysis.
Log messages with tag \DLlogger\ are excluded from the usual system log.

\subsection{Data archive}
The data archive maintained by \DLlogger\ contains two types of data files.
The first are called dump-files and are simply literal copies of the data
received from the \DTDL\ together with some diagnostics.
The second kind of data files use formats supported by libdatrwxx for outputs,
contains diagnostics too and might be more convenient for routine processing.
The second kind of data contains data in counts, not mm.
Both data archives contain files for completed days that provide the date in
the path name (as well as in the file's header).
The current data for uncompleted days is stored in files called
separate files, usually not indicating the date in the path name.
The latter are updated in periodic poll intervals.
Data is copied to the data storage host \Dstoragehost\ by \texttt{csback}
controlled by the \texttt{cron} daemon on \Dstoragehost.
The data archive can be provided to other hosts via network file system (NFS).
See Section~\ref{sec:prog:dllogger} for further comments.

\subsection{Timing}
Once in a day the data logger \DLlogger\ synchronizes the \DTDL\ to the system
clock of \Dloggerhost.
See Section~\ref{sec:prog:dllogger} for further comments on timing accuracy.
The system clock of \Dloggerhost\ must by disciplined by NTP (network time
protocol). 
To support easy monitoring of the system's state of health a shell script
\Dntpscript\ (Section~\ref{sec:prog:dlntpscript}) is executed once a day by
the cron deamon (see Section~\ref{sec:conf:crontab:logger}).
This script polls the state of NTP peers and writes it to the system log
stream with tag \DLlogger.
The system's time zone is selected to be UTC.
This way the time stamps in the system's log files are synchronous with the
times provided in the messages from \DLlogger.

\subsection{Data acquisition account}
The logger program \DLlogger\ runs in userspace of user \Dloggeruser\ on
\Dloggerhost. 
The home directory of \Dloggeruser\ is \Dloggeruserhome.
Binaries \DLlogger\ and \DLdirect\ are installed in
\Dloggeruserhome\texttt{/bin/linux}.
Shell scripts \Dlogscript, \Dlaunchscript, and \Dntpscript\ are installed in 
\Dloggeruserhome\texttt{/bin/scripts}.
The \DLlogger\ memory file, the data files, and logs of the launch script go
to \Dloggeruserhome\texttt{/DL1/data} and subdirectories therein.

\subsection{Data stream}
\DLlogger\ uses the following identifiers for the data stream in data files:
\vspace{-12pt}
\begin{description}
\parskip0pt
\itemsep0pt
\item[channel:] \Dchannel
\item[station:] is read from \DTDL
\item[instype:] is read from \DTDL
\end{description}
The channel ID is defined according to the SEED reference manual (Appendix~A):
\vspace{-12pt}
\begin{description}
\parskip0pt
\itemsep0pt
\item[Band code W:] Weather/Environmental
\item[Instrument Code R:] Rainfall
\item[Orientation Code:] unkown
\end{description}

%----------------------------------------------------------------------
\section{Data monitoring}
Data storage takes place on a second host (\Dstoragehost).
A third host (\Ddisplayhost) may be used for the purpose of monitoring and
data quality control. and data backup.
Log messages are made available through syslog in a local file
(Section~\ref{sec:conf:syslog:display}).

\subsection{Data graph monitor}
The data files are made available through \texttt{csback} and NFS.
Quality control plots are maintained through the monitor suite at BFO.
The processing and plotting mechanism is described elsewhere in a difference
document.

\subsection{Data file monitor}
For calibration purposes in particular a direct display of the current data
file can be useful.
The file can be displayed in a local terminal window by using the command
\begin{quote}
\texttt{less /data/BFO/DL1/data/thiesdl1/DL1current.thiesdl1}
\end{quote}
The display will be updated upon pressing the \texttt{R}-key.

\subsection{Message file monitor}
A permanent display of diagnostic messages can be maintained in a local
terminal window by the command
\begin{quote}
\texttt{less /var/log/dl1log}
\end{quote}
Pressing the \texttt{F}-key will activate automatic updates of the displayed
contents.
To use the file content navigation facilities of \texttt{less} press
\texttt{Ctrl-C}.
To return to automatic updates press the \texttt{F}-key again.

%----------------------------------------------------------------------
\end{multicols}
%\pagebreak
%======================================================================
\appendix
\section{Configuration}
\subsection{\texttt{crontab} on \Dloggerhost}
\label{sec:conf:crontab:logger}
\begin{verbatim}
# this is <crontab>
# ============================================================================
# crontab for netrunner
# ---------------------
# 
10 * * * * /home/dl1/bin/scripts/launchDL1logger.sh 1>>/dev/null 2>&1
50 0 * * * /home/dl1/bin/scripts/ntpreport.sh 1>>/dev/null 2>&1
# ----- END OF crontab ----- 
\end{verbatim}
%----------------------------------------------------------------------
%\subsection{\texttt{/etc/exports} on \Dloggerhost}
%\label{sec:conf:exports:logger}
%\begin{verbatim}
%[...]
%/home/dl1/data/DL1     *(ro,root_squash,sync,no_subtree_check)
%[...]
%\end{verbatim}
%----------------------------------------------------------------------
\subsection{\texttt{/etc/fstab} on \Ddisplayhost}
\label{sec:conf:fstab:display}
\begin{verbatim}
[...]
pinatubo:/data/BFO     /data/BFO    nfs     defaults,auto,ro 0 0
[...]
\end{verbatim}
%----------------------------------------------------------------------
\subsection{\texttt{syslog} configuration on \Dloggerhost}
The \texttt{syslog} daemon on \Dloggerhost\ should configured appropriately 
to maintain local log files and send messages to \Ddisplayhost\ and
\Dstoragehost.
Two local files are maintained. 
One of them collects all messages while the other only collects messages that
are not at level info.
If \texttt{rsyslog} is used, make sure that
\begin{verbatim}
[...]
#
# Include config files, that the admin provided? :
#
$IncludeConfig /etc/rsyslog.d/*.conf
[...]
\end{verbatim}
is present in \texttt{/etc/rsyslog.conf}.

\subsection{\texttt{/etc/syslog-ng/syslog-ng.conf} on \Dloggerhost}
\label{sec:conf:syslog:logger}
The following excerpt is from the syslog configuration file on the logger host
\Dloggerhost\ if \texttt{syslog-ng} is used.
\verbatiminput{DL1logger_syslog-ng_local.conf}

\subsection{\texttt{/etc/rsyslog.d/DL1logger\_rsyslog\_local.conf} on
\Dloggerhost}
The following file should be copied to \texttt{/etc/rsyslog.d}
on the logger host \Dloggerhost\ if \texttt{rsyslog} is used.
\verbatiminput{DL1logger_rsyslog_local.conf}
%----------------------------------------------------------------------
\subsection{\texttt{/etc/syslog-ng/syslog-ng.conf} on \Dstoragehost\ and
\Ddisplayhost}
\label{sec:conf:syslog:display}
The following excerpt is from the syslog configuration file on the 
storage host \Dstoragehost\  and display host \Ddisplayhost\ if
\texttt{syslog-ng} is used.
\verbatiminput{DL1logger_syslog-ng_remote.conf}

\subsection{\texttt{/etc/rsyslog.d/DL1logger\_rsyslog\_remote.conf} on
\Dstoragehost\ and \Ddisplayhost}
\label{sec:conf:rsyslog:display}
The following file should be copied to \texttt{/etc/rsyslog.d}
on the storage host \Dstoragehost\  and display host \Ddisplayhost\ if
\texttt{rsyslog} is used.
\verbatiminput{DL1logger_rsyslog_remote.conf}

%----------------------------------------------------------------------
\subsection{\texttt{/etc/logrotate.d/DL1\_logrotate.conf}}
\label{sec:conf:syslog:logrotate}
The log files will grow unlimited.
For this reason on each host maintaining a local copy of the log files,
a \texttt{logrotate} configuration file should be installed in 
\texttt{/etc/logrotate.d}:
\verbatiminput{DL1_logrotate.conf}

%----------------------------------------------------------------------
\section{Programs}
Source code is not documented here. 
You can find the source code at
\begin{quote}
\verb+https://git.scc.kit.edu/Seitosh/Seitosh+
\end{quote}
There you will find a link to
documentation for the DL1 software as well as a small
library for serial port access.
Use the instructive source code of well performing software to understand how
the \DTDL\ should be controlled via a serial line RS232 connection.

\subsection{\DLdirect}
\label{sec:prog:dldirect}
The program \DLdirect\ supports manual communication with the \DTDL.
Upon invocation it sends one command and displays the response on standard
output.
It might be unwise to use \DLdirect\ while an instance of \DLlogger\ is in
operation on the same serial port.
\verbatiminput{DL1direct.xxx}
%----------------------------------------------------------------------
\subsection{\DLlogger}
\label{sec:prog:dllogger}
The program \DLlogger\ is the actual data acquisition program.
\verbatiminput{DL1logger.xxx}
%----------------------------------------------------------------------
\subsection{\Dlogscript}
\label{sec:prog:dlogscript}
\verbatiminput{DL1logscript.xxx}
%----------------------------------------------------------------------
\subsection{\Dlaunchscript}
\label{sec:prog:dlaunchscript}
\verbatiminput{DL1launchscript.xxx}
%----------------------------------------------------------------------
\subsection{\Dntpscript}
\label{sec:prog:dlntpscript}
\verbatiminput{DL1ntpscript.xxx}
%----------------------------------------------------------------------


\end{document}

% ----- END OF DL1recording.tex ----- 
