% this is <gremlin.tex>
% ----------------------------------------------------------------------------
% $Id: gremlin.tex,v 1.7 2010/02/12 09:56:38 tforb Exp $
% 
% Copyright (c) 2006 by Thomas Forbriger (BFO Schiltach) 
% 
% Definition der Least-Squares Aufgabe, die von gremlin gel�st wird
% 
% REVISIONS and CHANGES 
%    24/08/2006   V1.0   Thomas Forbriger
%    17/12/2012   V1.1   x2c Darstellung im Programm ge"andert
% 
% ============================================================================
%
\documentclass{article}
\usepackage{ngerman}
\bibliographystyle{plain}
\usepackage{pslatex}
\usepackage{anysize}
\marginsize{20mm}{15mm}{15mm}{15mm}
\usepackage{amsmath}
\usepackage{graphicx}
\sloppy
\parindent0pt
\parskip10pt
%----------------------------------------------------------------------
\newcommand{\EHHz}{\text{Hz}}
\newcommand{\EHkm}{\text{km}}
\newcommand{\EHm}{\text{m}}
\newcommand{\EHs}{\text{s}}
\newcommand{\EHms}{\text{ms}}
\newcommand{\EHg}{\text{g}}
\newcommand{\EHccm}{\text{cm}^3}
\newcommand{\EHspkm}{\frac{\EHs}{\EHkm}}
\newcommand{\EHkmps}{\frac{\EHkm}{\EHs}}
\newcommand{\EHgpccm}{\frac{\EHg}{\EHccm}}
\newcommand{\EHTspkm}{\EHs/\EHkm}
\newcommand{\OFourier}[1]{\tilde{#1}}
\newcommand{\intii}{\int\limits_{-\infty}^{+\infty}}
\newcommand{\intni}{\int\limits_{0}^{+\infty}}
\DeclareMathOperator{\dd}{\text{d}}
\DeclareMathOperator{\Dd}{\Delta}
\newcommand{\Su}{u}
\newcommand{\SFu}{\OFourier{\Su}}
\newcommand{\SFupw}{\OFourier{\Su}^{\text{pw}}}
\newcommand{\Sus}{U}
\newcommand{\SFus}{\OFourier{\Sus}}
% use p_wave everywhere
\newcommand{\Spref}{p_{\text{wave}}}
%\newcommand{\Spwave}{p_{\text{wave}}}
\newcommand{\SG}{G}
\newcommand{\SGd}{\SG_\delta}
\newcommand{\SGpw}{\SG^{\text{SLS}}}
\newcommand{\SGbtr}{\SG^{\text{BTR}}}
\newcommand{\SGhop}{\SG^{\text{HOP}}}
\newcommand{\Sop}{\ensuremath{(\omega,p)}}
\newcommand{\Sokpl}{(\omega_k,p_l)}
\newcommand{\Sokpj}{(\omega_k,p_j)}
\newcommand{\Sor}{(\omega,r)}
\newcommand{\Sopr}{(\omega p r)}
\newcommand{\Seta}{\eta}
\newcommand{\Scbkj}{\{k,j\}}
\newcommand{\Scbkl}{\{k,l\}}
\newcommand{\Sseta}{_{\Seta}}
\newcommand{\SGpwop}{\SGpw\Sop}
\newcommand{\Stop}{\text{\ensuremath{\omega},\ensuremath{p}}}
\newcommand{\SslN}{\sum\limits_{l=1}^{N}}
\newcommand{\Sorl}{(\omega,r_l)}
\newcommand{\Soprl}{(\omega p r_l)}
\newcommand{\Soprrl}{(\omega \Spref r_l)}
\newcommand{\Skappa}{\beta}
\newcommand{\SHa}{H^{(1)}}
\newcommand{\SHb}{H^{(2)}}
\newcommand{\STcirc}[1]{\textcircled{\raisebox{-1pt}{#1}}}
\newcommand{\STa}{\STcirc{1}}
\newcommand{\STb}{\STcirc{2}}
\newcommand{\STc}{\STcirc{3}}
\newcommand{\SHM}{M}
\newcommand{\SHP}{\Theta}
\newcommand{\Snorm}[1]{\lVert #1 \rVert}
\newcommand{\Stnorm}[1]{\smash{\lVert\smash{#1}\rVert}}
\newcommand{\Stabs}[1]{\smash{\lvert\smash{#1}\rvert}}
\newcommand{\abs}[1]{\lvert#1\rvert}
% figures annotations
\newcommand{\Stred}{t_{\text{red}}}
\newcommand{\Svred}{v_{\text{red}}}
\newcommand{\Sxkappa}{\kappa}
% Hilbert space 
\newcommand{\HIvek}[1]{\mathbf{#1}}
\newcommand{\HIsprod}[2]{(#1,#2)}
\newcommand{\HImod}{\HIvek{m}}
\newcommand{\HIrep}[1]{\HIvek{g}_{#1}}
%
% elements of the inverse problem
\newcommand{\Scc}{^{\ast}}
\newcommand{\Mtransp}{\mathsf{T}}
\newcommand{\Madjung}{\dagger}
%\newcommand{\Mat}[1]{\ensuremath{\mathsf{#1}}}
%\newcommand{\Mat}[1]{\ensuremath{\mitbf{#1}}}
\newcommand{\Mat}[1]{\textsf{\textbf{#1}}}
\newcommand{\MatT}[1]{\ensuremath{\Mat{#1}^\Mtransp}}
\newcommand{\MatTcc}[1]{\ensuremath{\Mat{#1}^\Madjung}}
\newcommand{\Matinv}[1]{\ensuremath{\Mat{#1}^{\scriptscriptstyle{-1}}}}
\newcommand{\MatTinv}[1]{\ensuremath{\Mat{#1}^{-\Mtransp}}}
%\newcommand{\vek}[1]{\ensuremath{\vec{#1}}}
%\newcommand{\vek}[1]{\ensuremath{\mitbf{#1}}}
\newcommand{\vek}[1]{\textrm{\textbf{#1}}}
\DeclareMathOperator{\Mdiag}{diag}
\DeclareMathOperator{\Mtrace}{Tr}
\newcommand{\SSd}{d}
\newcommand{\SSe}{e}
\newcommand{\SSm}{m}
% resolution analysis: test parameter index
\newcommand{\SSPK}{L}
\newcommand{\SSmK}{m_{\SSPK}}
\newcommand{\SSDmK}{\Delta m_{\SSPK}}
\newcommand{\SSdm}{\delta m}
\newcommand{\SSs}{s}
% notice: equals \SSW now
\newcommand{\SSw}{W}
\newcommand{\SSsigma}{\sigma}
\newcommand{\SSr}{r}
\newcommand{\SSvs}{\ensuremath{v_{\text{S}}}}
\newcommand{\SSvp}{\ensuremath{v_{\text{P}}}}
\newcommand{\SSvph}{\ensuremath{v_{\text{Ph}}}}
% Poisson's ratio 
\newcommand{\SSnu}{\nu}
% resolution test damping factor
\newcommand{\SSdamp}{\lambda}
\newcommand{\SSD}{D}
\newcommand{\SSW}{W}
\newcommand{\SSS}{S}
\newcommand{\SSQ}{Q}
\newcommand{\SSQp}{\SSQ_{\text{p}}}
\newcommand{\SSQs}{\SSQ_{\text{s}}}
\newcommand{\SSK}{K}
\newcommand{\SSM}{M}
\newcommand{\SSMr}{R}
\newcommand{\SSMrp}{R^{\prime}}
\newcommand{\SSsrc}{S}
\newcommand{\SSsrcok}{\SSsrc(\omega_k)}
\newcommand{\Sve}{\vek{\SSe}}
\newcommand{\Svd}{\vek{\SSd}}
\newcommand{\Svs}{\vek{\SSs}}
\newcommand{\Svm}{\vek{\SSm}}
\newcommand{\Svmopt}{\vek{\SSm}_{\text{opt}}}
\newcommand{\Svdm}{\delta\vek{\SSm}}
\newcommand{\SvDm}{\Delta\vek{\SSm}}
\newcommand{\SvDmKmax}{\Delta\vek{\SSm}_{\SSPK\,\text{max}}}
\newcommand{\SMM}{\Mat{\SSM}}
\newcommand{\SMMr}{\Mat{\SSMr}}
\newcommand{\SMMrp}{\Mat{\SSMr}^{\prime}}
\newcommand{\SMD}{\Mat{\SSD}}
\newcommand{\SMDTcc}{\MatTcc{\SSD}}
\newcommand{\SMW}{\Mat{\SSW}}
\newcommand{\SMWT}{\MatT{\SSW}}
\newcommand{\SMS}{\Mat{\SSS}}
\newcommand{\SMST}{\MatT{\SSS}}
\newcommand{\Sesq}{E^2}
\newcommand{\Sesqmax}{E^2_{\text{max}}}
\newcommand{\Smifi}{\chi^2}
%----------------------------------------------------------------------
\newcommand{\Sbalance}{\beta}
\newcommand{\SnormG}{n^{(G)}}
\newcommand{\SnormT}{n^{(T)}}
\newcommand{\SweightG}{w^{(G)}}
\newcommand{\SweightT}{w^{(T)}}
\newcommand{\STerror}{e^{(T)}}
\newcommand{\SGerror}{e^{(G)}}
\newcommand{\Snu}{\nu}
%----------------------------------------------------------------------
\newcommand{\Szb}{\ensuremath{z_{\text{b}}}}
\newcommand{\Szt}{\ensuremath{z_{\text{t}}}}
\newcommand{\Szm}{\ensuremath{z_{\text{m}}}}
\newcommand{\Sxt}{\ensuremath{x_{\text{t}}}}
\newcommand{\Sxb}{\ensuremath{x_{\text{b}}}}
%----------------------------------------------------------------------
% Befehl
\newcommand{\B}[1]{\texttt{#1}}
% Programm
\newcommand{\Prog}[1]{\texttt{#1}}
\newcommand{\Func}[1]{\texttt{#1}}
\newcommand{\Rdistel}{F2001}
\newcommand{\RpaperII}{F2003}
%----------------------------------------------------------------------
\begin{document}
\title{Inversion von Wellenfeldspektren mit \texttt{gremlin}}
\author{Thomas Forbriger (BFO, Schiltach)}
\date{\today \\ 
{\tiny\texttt{
  $ $Date: 2010/02/12 09:56:38 $ $
  $ $Revision: 1.7 $ $
}}}
\twocolumn[\maketitle]
%======================================================================
\section{Formale Beschreibung}
Die Arbeit von Forbriger \cite{forbriger2001} beschreibt nicht genau die
Funktionsweise von \Prog{gremlin}, sondern geht dar"uber hinaus.
An einigen Stellen, k"onnen die Ausf"uhrungen aber unmittelbar auf
\Prog{gremlin} "ubertragen werden.
Eine h"ohere "Uberstimmung zwischen \Prog{gremlin} und den theoretischen
Formulierungen besteht im Artikel von Forbriger \cite{forbriger2003b}.
Im Folgenden bezeichnet \Rdistel\ die Dissertation von Forbriger
\cite{forbriger2001} und \RpaperII\ die Publikation von Forbriger
\cite{forbriger2003b}.
Die Notationen in beiden Publikationen ist leicht unterschiedlich.

\subsection{Das Least-Squares System}
Das Programm \texttt{gremlin} l"ost das Least-Squares Problem das in der Gl.\
(4) von \RpaperII\ als lineares Gleichungssystem
\begin{multline}
\Re\left\{\bigl(\SMDTcc\SMWT\SMW\SMD\bigr)+
          \SSdamp\SMST\SMS\right\}\,\Svdm_n\\
   = \Re\left\{\SMDTcc\SMWT\SMW\,\bigl(\Svd-\Svs(\Svm_n)\bigr)\right\}
\label{eq:LSQ:system}
\end{multline}
definiert ist.
Dabei bezeichnet $\Svd$ den Vektor der Messdaten mit den weiter unten in
Gl.~\eqref{eq:gdata} und Gl.~\eqref{eq:tdata} definierten Zuordnungen zu
$\Sop$-Koeffizienten aus der Analyse mit \Prog{greda} und den mit
\Prog{refract} abgelesenen Laufzeiten der Ersteins"atze.
Die synthetisch vorhergesagten Messdaten werden mit $\Svs(\Svm_n)$ bezeichnet,
wobei
$\Svm_n$ die Beschreibung des Untergrundmodells nach der $n$ten Iteration ist.
Entsprechend ist $\Svm_0$ das Startmodell der iterativen Inversion, das aus
anderen Informationen (z.B.\ Laufzeitinversion) gewonnen werden muss.
In der Matrix $\SMW$ stehen Gewichte f"ur die Messdaten.
Diese sind weiter unten
in den Gln.~\eqref{eq:gweight} und \eqref{eq:tweight} definiert.
$\SMWT$ bezeichnet die Transponierte von $\SMW$.
Hier ist $\SMWT=\SMW$, weil $\SMW$ in \Prog{gremlin} eine Diagonalmatrix ist.
$\SMD$ ist die Matrix der partiellen Ableitungen und $\SMDTcc$ die dazu
komplex konjugiert transponierte Matrix.
Die Berechnung der partiellen Ableitungen "uber Differenzenquotienten wird in
Kapitel~IV.1.3 von \Rdistel\ beschrieben.
In der Diagonalmatrix $\SMS$ und $\SMST$ stehen die Kehrwerte der sogenannten
Suchbereiche wie weiter unten in Gl.~\eqref{eq:searchrange} definiert.
$\Re$ bedeutet, dass der Realteil genommen wird.
$\SSdamp$ ist ein Regularisierungs- oder Stabilisierungsparameter der
Inversion.
Er wird weiter unten in Gl.~\eqref{eq:def:nu} definiert.

Die L"osung des Gleichungssystems \eqref{eq:LSQ:system} ergibt den Vektor der
Modell"anderungen $\Svdm_n$.
Das n"achste (verbesserte) Modell nach der $n$ten Iteration ist
\begin{equation}
\Svm_{n+1}=\Svm_n+\Svdm_n.
\label{eq:modelupdate}
\end{equation}

\subsection{Behandlung der Daten}
\subsubsection{Misfit}
Der quadratische Anpassungsfehler
\begin{equation}
    \Smifi(\Svm)=\abs{\SMW\,\bigl(\Svd-\Svs(\Svm)\bigr)}^2,
\end{equation}
der durch die Inversion minimiert werden soll, wird auch als \emph{Misfit}
bezeichnet.
Bei gegebenem Startmodell $\Svm_0$ kann der Misfit 
\begin{equation}
\Smifi(\Snu)=\Smifi(\Svm_1(\Snu))
\label{eq:mifi:nu}
\end{equation}
als Funktion des in Gl.~\eqref{eq:stabfacnu} definierten
Stabilisierungsfaktors $\Snu$ aufgefasst werden, da das Modell
\begin{equation}
\Svm_{1}(\Snu)=\Svm_0+\Svdm_0(\Snu)
\end{equation}
von $\Snu$ abh"angt, wenn $\Svdm_0(\Snu)$ durch das L"osen des
Gleichungssystems \eqref{eq:LSQ:system} f"ur ein
\begin{equation}
\SSdamp=\Snu/N_m
\label{eq:stabfacnu}
\end{equation}
gem"a"s Gl.~\eqref{eq:def:nu} gefunden wird.

\subsubsection{Gewichtung}
Vom Programm wird ein $\Sop$-Spektrum mit $N_f$ Frequenzen und $N_p$
Langsamkeiten in das Feld 
\begin{equation}
G_{kl} \quad\text{mit}\quad k=1\ldots N_p, l=1\ldots N_f
\end{equation}
eingelesen.
Beim Einlesen wird ein Feld
\begin{equation}
\SweightG_{kl}=1 \quad\text{mit}\quad k=1\ldots N_p, l=1\ldots N_f
\end{equation}
gesetzt, das im Bedarfsfall Gewichte f"ur die einzelnen $\Sop$-Koeffizienten
enthalten kann.
In ein zweites Feld
\begin{equation}
T_{k} \quad\text{mit}\quad k=1\ldots N_t
\end{equation}
werden $N_t$ Laufzeiten eingelesen.
Beim Einlesen wird ein Feld
\begin{equation}
\SweightT_{k}=1 \quad\text{mit}\quad k=1\ldots N_t
\end{equation}
gesetzt, dass im Bedarfsfall Gewichte f"ur die einzelnen Laufzeiten 
enthalten kann.
Dem Datenvektor $\Svd$ werden die eingelesenen Felder gem"a"s
\begin{align}
d_{l+(k-1)N_f}&=G_{kl} & \text{f"ur }& k=1\ldots N_p, l=1\ldots N_f\\
\label{eq:gdata}
\intertext{und}
d_{(N_{f}N_p)+k}&=T_k & \text{f"ur }& k=1\ldots N_t
\label{eq:tdata}
\end{align}
zugewiesen.
Die Gewichtungsmatrix $\SMW=\Mdiag(W_k)$ besteht aus den Elementen
\begin{gather}
\label{eq:gweight}
W_{l+(k-1)N_f}=
  \sqrt{\frac{\Sbalance\; (\SweightG_{kl})^2}{\SnormG {\SGerror}^2}}
  \quad\text{f"ur } k=1\ldots N_p, l=1\ldots N_f\\
\intertext{mit}
\SnormG=\sum\limits_{kl}(\SweightG_{kl})^2\\
\intertext{und}
\label{eq:tweight}
W_{(N_{f}N_p)+k}=
  \sqrt{\frac{(1-\Sbalance)\,(\SweightT_{k})^2}{\SnormT {\STerror}^2}}
 \quad\text{f"ur } k=1\ldots N_t\\
\intertext{mit}
\SnormT=\sum\limits_{k}(\SweightT_{k})^2.
\end{gather}
Dabei haben die Gr"o"sen in \texttt{gremlin} folgende Bezeichnungen:\\
\begin{tabular}{lll}
$\Sbalance$ & \texttt{bala} & balance \\
$\STerror$ & \texttt{terr} & travel time error \\
$\SGerror$ & \texttt{gerr} & green amplitude error \\
\end{tabular}

Weitere Hinweise zur Wahl der Gewichte $W_k$ f"ur die Daten findet man 
in Abschnitt~2.2 von \RpaperII\ und Kapitel~V.1.4.1 von \Rdistel.

\subsection{Behandlung der Modellparameter}
\subsubsection{Terminologie}
\label{sec:mod:term}
  \paragraph{Freie Modellparameter:}
  Es wird zwischen freien und nicht-freien Modellparameter unterschieden.
  Die Modellparameter sind alle Parameter, die das Untergrundmodell
  beschreiben. 
  Dazu geh"oren die Polynomkoeffizienten der seismischen Gesschwindigkeiten,
  der Dichte und der G"utefaktoren, aber auch die Werte der Tiefe der
  Unterkante der einzelnen Polynomsektionen.
  Nur ein Teil dieser Parameter wird in der Inversion ver"andert.
  Welcher Teil ver"andert wird, muss vom Benutzer ausgew"ahlt werden. 
  Diesen sogenannten \emph{freien Modellparametern} m"ussen Suchbereiche
  zugewiesen werden.
  Nur diese Parameter werden in den Modellvektor $\Svm$ "ubertragen.
  Alle anderen Parameter des Modells k"onnen durch die Inversion nicht
  ver"andert werden, behalten damit ihren Startwert und sind in diesem Sinne
  \emph{nicht-frei}.

  \paragraph{Named model parameters:} Das Programm unterscheidet zwischen
  \emph{namend model parameters} und \emph{anonymous model parameters}.
  \emph{namend model parameters} sind die Modellparameter in einer
  Beschreibung der Untergrundstruktur, in der klar zwischen physikalischen
  Parameter wie \SSvp, \SSvs, Dichte, etc.\ unterschieden wird.
  Die freien Modellparameter, die im Modellvektor $\Svm$ behandelt werden,
  sind \emph{anonymous model parameters} in dem Sinne, dass deren
  physikalische Natur f"ur den Inversionsalgorithmus unerheblich ist und die
  Gr"o"sen nach Gewichtung mit den Suchbereichen einheitenlos werden.
  Der Befehl \B{tpc} zeigt die Zuordnung der anonymen Modellparameter an.

  \paragraph{Suchbereiche:} Die Suchbereiche werden im Programm auch
  \emph{model parameter weights} genannt.

\subsubsection{Suchbereiche und Regularisierung}
Alle Modellparameter, die in die Inversion einbezogen werden (Tiefe von
Diskontinuit"aten, seismische Geschwindigkeiten, etc.) werden in einen Vektor
$\Svm$ eingetragen.
Die Suchbereiche $r_k$, die mit dem Kommando \texttt{spc} gesetzt werden,
definieren entsprechend der Gl.\ (9) in \RpaperII\
die Matrix 
\begin{equation}
\SMS=\Mdiag\left(\frac{1}{r_k}\right)
\quad\text{mit}\quad k=1\ldots N_m,
\label{eq:searchrange}
\end{equation}
wobei $N_m$ die Anzahl der in der Inversion aktivierten Modellparameter ist.
Der D"ampfungsparameter in Gl.\ \eqref{eq:LSQ:system} ist
\begin{equation}
\label{eq:def:nu}
\SSdamp=\Snu/N_m.
\end{equation}
Dabei hat die Gr"o"se $\Snu$ in \texttt{gremlin} folgende Bezeichnung:
\begin{tabular}{lll}
$\Snu$ & \texttt{nu} & nu \\
\end{tabular}

\subsection{Parametrisierung}

Grunds"atzliche "Uberlegungen zur Entwicklung des Untergrundmodells nach
Basisfunktionen werden in Kapitel~IV.1.2 von \Rdistel\ angestellt.
In Kapitel~V.2.1 von \Rdistel\ wird die Orthogonalit"atsforderung f"ur die
Basisfunktionen der Modellparametriesierung beschrieben.
Diese "Uberlegungen wurden aber in \Prog{gremlin} nicht in voller
Allgemeinheit umgesetzt.
Die Parameter des Untergrunds, also die Kompressionswellengeschwindigkeit
$\SSvp$, die Scherwellengeschwindigkeit $\SSvs$, die Dichte $\rho$ und die
G"utefaktoren f"ur die Geschwindigkeiten sind abschnittsweise (in Sektionen)
nach Polynomen 2ten Grades entwickelt.
Sie sind damit Funktionen der Tiefe $z$, die nach unten positiv gerechnet
wird.
Die Inversionsparameter sind die Polynomkoeffizienten und die Tiefen in denen
eine Sektion endet und die n"achste beginnt.
Die Bedeutung der Polynomkoeffizenten in den Modelldateien wird in
Kapitel~A.2.2 im Anhang von \Rdistel\ definiert.
Diese entprechen aber nicht den Polynomkoeffizienten, die in \Prog{gremlin}
als Inversionsparameter im Vektor $\Svdm$ verwendet werden. 
Die Inversionsparameter werden weiter unten in
Abschnitt~\ref{sec:def:polynomial} definiert.

\subsubsection{Terminologie}
  Bei den Modelldateien wird zwischen \emph{polynomial
  models} und \emph{discrete models} unterschieden:
  \begin{description}
  \item[polynomial model:] Darstellung des Untergrundmodells in Form von
  Sektionen innerhalb derer der Verlauf der Parameter durch Polynome
  beschrieben wird.
  In dieser Parametriesierung wird das Inversionsproblem gel"ost.
  Die Modelldateien, die von \Prog{gremlin} gelesen und geschrieben werden,
  benutzen diese Darstellung.
  \item[discrete model:] Darstellung des Untergrundmodells in Form homogener
  Schichten.
  Mit dieser Parametriesierung arbeitet die Refelktivit"atsmethode zur
  Berechnung synthetischer Daten.
  Innerhalb von \Prog{gremlin} muss das Polynommodell vor jeder
  Vorw"artsrechnung diskretisiert werden.
  Das Programm \Prog{mocon} ist in der Lage Dateien beider Formen der
  Modellbeschreibung in die jeweils andere zu konvertieren.
  \end{description}

\subsubsection{Definition der Polynomparameter}
\label{sec:def:polynomial}
  Sei der Wert eines Parameters $p$ durch ein Polynom innerhalb einer Sektion
  mit der Oberkante in der Tiefe $\Szt$ und der Unterkante bei $\Szb$ gegeben.
  Die $z$-Koordinate wird dabei nach unten positiv gerechnet.
  Die Sektion hat die M"achtigkeit
  \begin{equation}
  h=\Szb-\Szt
  \end{equation}
  und ihre Mitte liegt bei
  \begin{equation}
  \Szm=\frac{1}{2}\,\left(\Szb+\Szt\right).
  \end{equation}
  Das Polynom wir bez"uglich der Mitte \Szm\ mit den Parametern
  \begin{equation}
  \Sxt=\Szt-\Szm\qquad\text{und}\qquad
  \Sxb=\Szb-\Szm
  \end{equation}
  definiert.
  Der Parameter in der Tiefe $z=x+\Szm$ ist also
  \begin{equation}
  p(x)=p_0+p_1\,x+p_2\,x^2.
  \end{equation}
  Das entspricht der Darstellung in Kapitel~A.2.2 im Anhang von \Rdistel.
  Die Funktion \Func{mod\underbar{ }chop} wertet die Polynomparameter in
  dieser Weise aus.

  Die \emph{anonymous model parameters} (siehe Abschnitt~\ref{sec:mod:term}) 
  im Vektor $\Svdm$
  beschreiben "Anderungen, die an die Koeffizienten $p_0$, $p_1$ und $p_2$
  angebracht werden.
  Sind $m_0$, $m_1$ und $m_2$ Elemente des Vektors $\Svdm$, so ergeben sich
  die neuen Modellwerte mit
  \begin{align}
  p_0^{\prime}&=p_0+m_0-m_2\,\frac{h^2}{12}, \\
  p_1^{\prime}&=p_1+m_1 \quad\text{und} \\
  p_2^{\prime}&=p_2+m_2 \quad\text{mit}\\
  p^{\prime}(x)&=p_0^{\prime}+p_1^{\prime}\,x+p_2^{\prime}\,x^2.
  \end{align}
  Basisfunktionen im Sinne von Kapitel~V.2.1 in \Rdistel\ sind damit
  \begin{align}
  b_0(z)&=1, \\
  b_1(z)&=z-\Szm\quad\text{und} \\
  b_2(z)&=(z-\Szm)^2-\frac{h^2}{12}. 
  \end{align}

  Die Funktion \Func{mod\underbar{ }parcor} setzt diese Zuordnung um.
  Der Vorteil dieser Zuordnung besteht darin, dass die "Anderung des
  Mittelwerts 
  \begin{equation}
  \overline{\Delta p}=\frac{1}{h}\int\limits_{\Sxt}^{\Sxb} 
  p^{\prime}(x)-p(x)\,\text{d}x=m_0
  \end{equation}
  des Parameters in der Sektion wirklich nur von $m_0$ abh"angt.
  Au"serdem h"angt die "Anderung des mittleren Gradienten
  \begin{equation}
  \overline{\Delta \delta p}=\frac{1}{h}\int\limits_{\Sxt}^{\Sxb} 
  \frac{\text{d}}{\text{d}x}\left(p^{\prime}(x)-p(x)\right)\,\text{d}x=m_1
  \end{equation}
  des Parameters in der Sektion wirklich nur von $m_1$ ab.

\subsubsection{Dateiformat}
Das Dateiformat f"ur die Polynommodelle wird in Tabelle~\ref{tab:model:format}
beschrieben.
\begin{table*}
\begin{centering}
\begin{verbatim}
Wie final0498.p.mod aber einfacheres Qp + physikalisch sinnvoller Halbraum

ModVersion 2        
    3              <-- number of sections

    Vp             Vs             density        Qp             Qs

      2.6643       <-- bottom of section     1 / polynomial expansion:
           3              3              1              1              1

  .374036884     .237842164    1.600000000        11.9263        32.1174
  .132729748     .071226151     .000000000         0.0000          .0000
 -.038238648    -.041646658     .000000000         0.0000          .0000

     16.3814       <-- bottom of section     2 / polynomial expansion:
          -2             -2              1              1              1

  .601509958     .359275030    1.600000000        50.7330        32.1174
  .017280088     .014646841     .000000000          .0000          .0000
  .000000000     .000000000     .000000000          .0000          .0000

     25.0000       <-- bottom of section     3 / polynomial expansion:
           1              1              1              1              1

 3.702443237    2.150750911    2.300000000       200.0000       100.0000
  .000000000     .000000000     .000000000          .0000          .0000
  .000000000     .000000000     .000000000          .0000          .0000
\end{verbatim}\par
\end{centering}
\caption{Die Datei \texttt{biet000601.p.mod.dat} f"ur das Modell
\textsc{Bietigheim} in Tabelle~VI.1 in \Rdistel.
Das Modell besteht aus drei Sektionen.
Zu Beginn jeder Sektion wird die Tiefe der Unterkante der Sektion definiert.
Unter der Unterkante der untersten Sektion wird das Modell implizit mit einem
konstanten Wert der Parameter an der Unterkante der untersten Sektion
fortgesetzt.
Die f"unf physikalischen Parameter (P-Wellengeschwindigkeit $\SSvp$,
S-Wellengeschwindigkeit $\SSvs$, Dichte $\rho$, G"ute $Q_\text{P}$ f"ur
$\SSvp$ und G"ute $Q_\text{S}$ f"ur $\SSvs$) sind mit ihren drei
Polynomkoeffizienten pro Sektion in Spalten nebeneinander angeordnet.
Zu jedem Parameter in jeder Sektion wird angegeben, wie viele der
Polynomkoeffizienten verwendet werden. 
1 bedeutet ein Polynom 0ten Grades, also eine homogene Schicht.
2 bedeutet ein Polynom 1ten Grades, also eine Gradientenschciht.
3 bedeutet ein Polynom 2ten Grades.
Die Polynomkoeffizienten in jedem Block sind von oben nach unten $p_0$, $p_1$
und $p_2$.
Ein Minus vor der Anzahl der Polynomkoeffizienten bedeutet, dass der
Mittelwert der jeweiligen Sektion (Koeffizient $p_0$) automatisch so justiert
wird, dass an der Oberkante der Sektion keine Diskontinuit"at auftritt.
In der Inversion folgt dann der Mittelwert dieser Sektion den "Anderungen in
der dar"uber liegenden Sektion.
In \Prog{gremlin} wird dies durch die Angabe \texttt{follow} in der
Parameterzuordnung angezeigt.
}
\label{tab:model:format}
\end{table*}
%======================================================================
\section{Programmfunktionen}
\subsection{Vorgehensweise}
  \begin{enumerate}
  \item
  Vor dem Aufruf von \Prog{gremlin} sollte ein Protokollmitschrieb mit dem
  Unix-Befehl \B{script} eingeschaltet werden.
  \item
  Nach Aufruf des Programms und dem Einlesen der Laufzeitdaten, des
  \Sop-Spektrums, des Startmodells und der Programmparameter wird zun"achst
  der Kontrollbildschirm mit \B{dda} aufgerufen und damit das Startmodell
  "uberpr"uft.
  Dies muss in jedem Fall der erste aufgerufene Befehl sein, da dieser Befehl
  einige Indexwerte initialisiert, die von anderen Befehlen ben"otigt werden.
  \item
  Mit den Befehlen \B{dgr} und \B{dtt} k"onnen die eingelesenen Messdaten
  dargstellt und die Auswahl der Messdaten (Frequenz-, Langsamkeits- und
  Offsetbereich) "uberpr"uft werden.
  \item
  Mit dem Befehl \B{spa} wird in das Untermen"u zur Kontrolle der
  Steuerparameter gewechselt.
  Hier k"onnen Kontrollparameter (Gewichtung der Messdaten, Modus zur
  Gewinnung der Fouriertransformation des Quellwavelets, etc.) justiert
  werden.
  \item
  Mit dem Befehl \B{verb} wird in das Untermen"u zur Steuerung der
  Kontrollausgaben gewechselt.
  Hier ist insbesondere die Option \B{mod} interessant, die daf"ur sorgt, dass
  die Zwischenergebnisse der Inversion in der Bildschirmausgabe protokolliert
  werden.
  \item 
  Bevor einen Inversion durchgef"uhrt werden kann, m"ussen die Suchbereiche
  f"ur die freien Modellparameter gesetzt werden.
  Dies geschieht in einem Untermen"u, das mit \B{spc} erreicht wird.
  Modellparameter, f"ur die kein Suchbereich gesetzt ist, geh"oren nicht zu
  den \emph{freien Modellparametern} und bleiben in der Inversion
  unver"andert.
  Mit dem Befehl \B{wread} im Men"u \B{file} k"onnen Suchbereiche aus einer
  Datei gelesen werden, die das gleiche Format hat, wie die
  Polynommodelldateien.
  \item
  Nachdem die Suchbereich gesetzt sind, kann gepr"uft werden, ob die Gr"o"se
  der Suchbereiche und die Auswahl der freien Modellparameter ein robustes
  Inversionsproblem definiert.
  Dies geschieht mit dem Befehl \B{x2c} im Hauptprogramm (siehe
  Abschnitt~\ref{sec:cmd:x2c}).
  Dieser Befehl sollte vor jeder Inversion aufgerufen werden, um den
  Stabilisierungsfaktor $\Snu$ aus Gleichung \eqref{eq:def:nu} zu justieren.
  \item Danach kann eine iterative Inversion aufgerufen werden. 
  Dies geschieht meistens mit dem Befehl \B{ofi} (siehe
  Abschnitt~\ref{sec:cmd:ofi}). 
  Danach werden eventuell Steuerparameter korrigiert und eine neue Inversion
  aufgerufen.
  Diese Vorgehensweise der doppelten Iteration wird in Abschnitt~2.7.2 in
  \RpaperII\ beschrieben.
  \item 
  Wenn mit der iterativen Inversion kein weiterer Fortschritt erzielt werden
  kann, oder die Inversion in die Irre gegangen ist, kann es sinvvol sein,
  manuelle "Anderungen am Modell anzubringen und diese auszuprobieren.
  Mit dem Befehl \B{med} wird der \Prog{vim}-Editor mit dem aktuellen
  Referenzmodell aufgerufen.
  Das Modell kann manuell ge"andert und gespeichert werden und wird nach dem
  Verlassen des Editors automatisch von \Prog{gremlin} wieder eingelesen.
  \item 
  Ist das Ergebnismodell nach mehreren Iterationen erfolgreich, kann das
  Ergebnismodell im Untermen"u \B{file} mit dem Befehl \B{save} gespeichert
  werden.
  \item 
  Manchmal ist es hilfreich den Beitrag der partiellen Ableitungen
  abzusch"atzen, um zu "uberpr"ufen f"ur welchen Parameter die Daten am
  sensitivsten sind.
  Dies ist im Untermen"u \B{reso} mit dem Befehl \B{parder} m"oglich.
  \end{enumerate}
\subsection{Hauptmen"u}
\subsubsection{Befehl \B{x2c}}
\label{sec:cmd:x2c}
  Mit diesem Befehl wird gepr"uft, ob die Suchbereiche und die freien
  Parameter robust gew"ahlt sind.
  Sind deren Werte gut gew"ahlt, so sollte der Misfit $\Smifi(\Snu)$ als
  Funktion des Stabilisierungsparameters $\Snu$ gem"a"s Gl.~\eqref{eq:mifi:nu}
  mit kleiner werdendem $\Snu$ abnehmen.
  Zur Kontrolle berechnet der Befehl \B{x2c}
  den Wert von $\Smifi$ f"ur mehrere Werte von
  $\Snu$ auf einer logarithmischen Skala und stellt sie "uber
  $\Snu$ dar (gro"se D"ampfung links, kleine D"ampfung rechts).
  Sind die Parameter robust eingestellt, muss der Wert der Kurve im Plot nach
  rechts abnehmen, also mit abnehmendem Wert von $\Snu$.
  Die Berechnung der Kurve erfolgt vergleichsweise rasch, weil dazu nur einmal
  partielle Ableitungen berechnet werden m"ussen.
  Der Aufruf erfolgt mit
  \begin{quote}
    \B{x2c numin,numax,npts}
  \end{quote}
  \begin{description}
  \item[\texttt{numin}:]
  Der kleinste Wert f"ur $\Snu$ (rechter Rand des Plots)
  \item[\texttt{numax}:]
  Der gr"o"ste Wert f"ur $\Snu$ (linker Rand des Plots)
  \item[\texttt{npts}:]
  Die Anzahl von Werten f"ur die $\Smifi(\Snu)$ berechnet werden
  soll.
  \end{description}

\subsubsection{Befehl \B{ofi}}
\label{sec:cmd:ofi}
  Haben sich die Suchbereiche beim Aufruf von \B{x2c} als robust erwiesen,
  kann mit dem Befehl \B{ofi} eine iterative Inversion durchgef"uhrt werden.
  Der Aufruf erfolgt mit
  \begin{quote}
    \B{ofi nu,lim,step,mode}
  \end{quote}
  \begin{description}
  \item[\texttt{nu}:] 
  Dies ist der Wert des Stabilisierungsparameters $\Snu$
  mit dem das Gleichungssystem gel"ost werden soll. 
  Dieser sollte noch im abfallenden Teil der \B{x2c}-Kurve liegen.
  \item[\texttt{lim}:] 
  Grenzwert f"ur $\Smifi$. Wird dieser Wert unterschritten, so wird die
  Inversion als erfolgreich betrachtet und die Iteration abgebrochen.
  \item[\texttt{step}:] 
  Maximale Anzahl von Iterationsschritten, die durchgef"uhrt werden sollen.
  \item[\texttt{mode}:] 
  Anzeigemodus w"ahren der Iteration. 
  Ich empfehle, diesen Wert auf 2 zu setzen.
  \end{description}

%======================================================================
\section{Hilfsprogramme}
  Folgende Hilfsprogramme stehen zur Verf"ugung:
  \begin{description}
  \item[\Prog{clemo}] legt entweder ein leeres Template f"ur ein Polynommodell
  mit einer vorgegebenen Anzahl Sektionen an oder korrigiert die Formatierung
  eines vorhandenen Modells.
  \item[\Prog{grereso}] berechnet die Aufl"osungsmatrix f"ur die
  \Prog{greda}-Inversion wie in Kapitel~III.3 in \Rdistel\ beschrieben.
  \item[\Prog{moche}] pr"uft die Werte in einer Modellbeschreibung auf
  physikalische Sinnhaftigkeit der Parameter (siehe auch Abschnitt~2.5 in
  \RpaperII).
  Dazu werden aus den Geschwindigkeiten elastische Moduln und die
  dazugeh"origen G"ute-Werte, sowie die Poisson-Zahl berechnet.
  Siehe dazu auch die Bemerkungen In Kapitel~II.2.2 in \Rdistel.
  Beachte, dass die Definition der G"ute f"ur die Scherwellengeschwindigkeit
  in Gl.~(II.20) in \Rdistel\ nicht mit der
  Definition in \Prog{gremlin} "ubereinstimmt.
  Die Definition seismischer Geschwindigkeiten und dazugeh"origer G"utewerte
  wird von Forbriger und Friederich \cite{forbriger2005} diskutiert.
  \item[\Prog{mocon}] dient zur Konvertierung von Polynommodellen in Modelle
  homogener Schichten und umgekehrt. 
  \item[\Prog{mop}] stellt Polynommodelle graphisch dar.
  \item[\Prog{mops}] dient zur Erzeugung von Startmodellen. 
  Wurde ein Modell der P-Wellengeschwindigkeit \SSvp\ aus der Laufzeitkurve der
  refraktierten Wellen gewonnen, so kann mit \Prog{mops} ein
  korrespondierendes Modell der S-Wellengeschwindigkeit
  \begin{equation}
  \SSvs=\frac{\SSvp}{\sqrt{3}}
  \end{equation}
  erzeugt werden.
  \item[\Prog{moshift}] dient zu Parameterstudien.
  Die Parameter eines Polynommodells k"onnen um einen gegebenen Wert
  ver"andert werden.
  Dieser Wert wird in einer getrennten Datei definiert (\texttt{shiftfile})
  und mit einem skalaren Faktor (\texttt{v}) multipliziert zum Eingangsmodell
  addiert.
  Auf dieser Weise kann relativ einfach ein Gridsearch durchgef"uhrt werden.
  \item[\Prog{motab}] erzeugt eine ASCII-Tabelle aus einem Polynommodell.
  Dies dient zur Darstellung von Polynommodellen in externen Plotprogrammen.
  \item[\Prog{moval}] dient zur Abfrage der Werte der Modellparameter in einer
  gegebenen Tiefe f"ur ein vorgegebenes Polynommodell.
  \item[\Prog{pmotra}] stellt das \SSvp-Modell eines Polynommodells zusammen
  mit eingelesenen Ankunftszeiten der P-Wellen und der theoretischen
  Laufzeitkurve der refraktieren P-Wellen graphisch  dar.
  \item[\Prog{polex}] dient zur Berechnung der geeigneten Polynomkoeffizienten
  bei gegebenen Werten der Modellparameter in verschiedenen Tiefen.
  \item[\Prog{splimo}] erzeugt ein Polynommodell aus einen gegebenen
  Polynomodell, wobei im erzeugten Polynommodell eine Sektion des
  Eingabemodells in zwei Sektionen des Ausgabemodells zerlegt wird.
  Dies erlaubt einen nachtr"agliche Einbau von Diskontinuit"aten.
  \end{description}

\bibliography{gremlin}
\end{document}

% ----- END OF gremlin.tex ----- 
